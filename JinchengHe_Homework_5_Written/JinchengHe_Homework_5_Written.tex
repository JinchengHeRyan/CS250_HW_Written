\documentclass{article}
\usepackage[utf8]{inputenc}
\usepackage[english]{babel}
\usepackage[margin=1in]{geometry}

\usepackage{fancyhdr}
\usepackage{extramarks}
\usepackage{amsmath}
\usepackage{amsthm}
\usepackage{amsfonts}
\usepackage{tikz}
\usepackage[plain]{algorithm}
\usepackage{algpseudocode}
\usepackage{arydshln}
\usepackage{mathtools}
\usepackage{cases}
\usepackage{listings}
\usepackage[numbered]{mcode}
\usepackage{booktabs}
\usepackage{graphicx}
\usepackage{subfigure}

\usepackage{blindtext}
\usepackage{amssymb}
\usepackage{hyperref}
\hypersetup{
    colorlinks=true,
    linkcolor=blue,
    filecolor=magenta,
    urlcolor=cyan,
}

\urlstyle{same}

%\newtheorem{theorem}{Theorem}
\newtheorem{theorem}{Theorem}[section]
\newtheorem{corollary}{Corollary}[theorem]
\newtheorem{lemma}[theorem]{Lemma}
\theoremstyle{remark}
\newtheorem*{remark}{Remark}

\theoremstyle{definition}
\newtheorem{definition}{Definition}[section]

\title{CS 250 - Computer Architecture \\ Homework 5 Written}
\author{Jincheng He Email: jincheng.he@dukekunshan.edu.cn}
\date{December 13, 2021}

\begin{document}

    \maketitle


    \section{Q1. Cache policies}
    \textbf{Why are write-back caches usually also write-allocate?}

    The advantage of write-back caches is when writing to a block for multiple times, it can stay in cache, only need to be written to main memory when it is going to be replaced.
    So if the block is loaded on a write miss, if there is any subsequent writes, it can still stay in cache instead of going to the main memory, which can save time.


\end{document}