\documentclass{article}
\usepackage[utf8]{inputenc}
\usepackage[english]{babel}
\usepackage[margin=1in]{geometry}

\usepackage{fancyhdr}
\usepackage{extramarks}
\usepackage{amsmath}
\usepackage{amsthm}
\usepackage{amsfonts}
\usepackage{tikz}
\usepackage[plain]{algorithm}
\usepackage{algpseudocode}
\usepackage{arydshln}
\usepackage{mathtools}
\usepackage{cases}
\usepackage{listings}
\usepackage[numbered]{mcode}
\usepackage{booktabs}
\usepackage{graphicx}
\usepackage{subfigure}

\usepackage{blindtext}
\usepackage{amssymb}
\usepackage{hyperref}
\hypersetup{
    colorlinks=true,
    linkcolor=blue,
    filecolor=magenta,      
    urlcolor=cyan,
}

\urlstyle{same}

%\newtheorem{theorem}{Theorem}
\newtheorem{theorem}{Theorem}[section]
\newtheorem{corollary}{Corollary}[theorem]
\newtheorem{lemma}[theorem]{Lemma}
\theoremstyle{remark}
\newtheorem*{remark}{Remark}

\theoremstyle{definition}
\newtheorem{definition}{Definition}[section]

\title{MATH 304 - Numerical Analysis and Optimization \\ Project 1---Least Squares Regression}
\author{Jincheng He  Email: jincheng.he@dukekunshan.edu.cn}
%\date{October 2019}

\begin{document}

\maketitle
\begin{abstract}
    In this paper, we talk about Euler's famous solution to the Basel problem and his advantage over other mathematicians of the time.
\end{abstract}
\section{Introduction to the Basel Problem}
The Basel problem was originally posed by the Italian mathematician Pietro Mengoli in 1650. The problem was to find the value of the infinite sum
    \[\zeta(2)=\sum_{n=1}^{\infty}\frac{1}{n^2}.\]
Nearly 90 years later, it was famously solved by Leonhard Euler in 1735, solving this got Euler a lot of the fame he has today. Euler's solution was an intriguing value of  $\frac{\pi^2}{6}$. The problem to most mathematicians was the sum converged extremely slowly and they could not guess a good value of the sum to prove. But Euler had an advantage; he could approximate the sum better.
\section{How did Euler approximate the value of the sum?}

Other mathematicians attempts were to get the sum at small values and guess the answer. Here are the approximations of the sum;
\[n=10, \qquad\sum_{n=1}^{10}\frac{1}{n^2}=1.5497677311665408\]
\[n=100, \qquad  \sum_{n=1}^{100}\frac{1}{n^2}=1.6349839001848923\]
\[n=1000,\qquad\sum_{n=1}^{1000}\frac{1}{n^2}=1.6439345666815615\]
as you can see, the convergence is really bad, so Euler had a method for the acceleration of series (see \cite{Gos00}). To start, let's integrate the Taylor series for $\frac{-\ln(1-x)}{x}$,
\begin{align}
\int_{0}^{t}\frac{-\ln(1-x)}{x}dx &=\int_{0}^{t}\sum_{n=1}^{\infty}\frac{x^{n-1}}{n} dx \\
&= \sum_{n=1}^{\infty}\frac{x^{n}}{n^2}
\end{align}
Plugging in $t=1$ will give the Basel problem, so we have:
\begin{align} 
\sum_{n=1}^{\infty}\frac{1}{n^2}&=\int_{0}^{1} \frac{-\ln(1-x)}{x}dx\\
& = \int_{0}^{1/2}\frac{-\ln(1-x)}{x}dx+\int_{1/2}^{1}\frac{-\ln(1-x)}{x}dx
\end{align}
There is no easy way of evaluating this integral from the margin 0 to 1 to get the solution to the Basel Problem, so Euler did something different. Euler split the integral into 2 parts,
\begin{align} 
\sum_{n=1}^{\infty}\frac{1}{n^2}&=\int_{0}^{1/2}\frac{-\ln(1-x)}{x}dx+\int_{1/2}^{1}\frac{-\ln(1-x)}{x}dx\\&= \sum_{n=1}^{\infty}\frac{1}{n^22^n}+\int_{1/2}^{1}\frac{-\ln(1-x)}{x}dx.
\end{align}
Now, let $x=1-t$;
\[\zeta(2)= \sum_{n=1}^{\infty}\frac{1}{n^22^n}-\int_{0}^{1/2}\frac{\ln(x)}{1-x}dx\]
observe that $\frac{1}{1-x}$ is a geometric series. Plugging this in we have:
\[\zeta(2)= \sum_{n=1}^{\infty}\frac{1}{n^22^n}-\int_{0}^{1/2}\sum_{n=0}^{\infty}\ln(x)x^ndx\]
 we use a calculator to compute the integral:
\[\zeta(2)=\sum_{n=1}^{\infty}\frac{1}{n^22^n}-\sum_{n=1}^{\infty}\frac{1}{n}\left(\frac{1}{2}^{n}\ln \left(\frac{1}{2}\right)-\frac{\frac{1}{2}^{n}}{n}\right).\]
Simplifying a little bit, we have:
\[\zeta(2)=2\sum_{n=1}^{\infty}\frac{1}{n^22^n}+\ln(2)\sum_{n=1}^{\infty}\frac{1}{n2^n}\]
and using the power series for $-ln(1-x)$ with $x=\frac{1}{2}$,
\[-\ln(1-\frac{1}{2})=\ln(2)=\sum_{n=1}^{\infty}\frac{1}{n2^n}\]
plugging this in, we get;
\[\zeta(2)=\sum_{n=1}^{\infty}\frac{1}{n^22^n}+\ln(2)^2.\]
Because the terms in the sequence get closer and closer to 0 faster, it must converge to $\zeta(2)$ faster. Here are the sum of a few terms;
\[n=10, \zeta(2)=\sum_{n=1}^{10}\frac{1}{n^22^n}+\ln(2)^2=1.64492005167\]
\[n=100, \zeta(2)=\sum_{n=1}^{100}\frac{1}{n^22^n}+\ln(2)^2=1.64493406685\]
\[n=1000, \zeta(2)=\sum_{n=1}^{1000}\frac{1}{n^22^n}+\ln(2)^2=1.64493406685.\]
This converges so well that the first 12 digits are the same and 
\[\frac{\pi^2}{6}-(\sum_{n=1}^{1000}\frac{1}{n^22^n}+\ln(2))=-1.7739143e-12\]
a very small number. Euler was now very convinced that the sum was $\frac{\pi^2}{6}$, and if you know the answer to a sum, it is much easier to prove the value of the sum. 

\section{The proof of the Basel problem}
Euler's proof, unlike his previous analysis, was a lot simpler and less rigorous.
\begin{theorem}
The sum
\[\sum_{n=1}^{\infty}\frac{1}{n^2}=\frac{\pi^2}{6}\].

\end{theorem}
\begin{proof}
In order to prove this, we need the lemma:
\begin{lemma}
The sine function can be expressed as the following infinite product;
\[\sin(x)=x\prod_{n=1}^{\infty}(1-\frac{x^2}{\pi^2n^2}).
\]
\end{lemma}
\begin{proof}
If we wanted to have a product for $\sin(x)$, we can write in terms of it zero's. For example, if we wanted to write $x^2+x+2$ as a product, we would consider it's zeros: -2, and -1, then
\[x^2+x+2=c(x+2)(x+1)\]
where c is a constant decided by $f(0)$. We can do this with $\frac{\sin(x)}{x}$ (it turns out to be easier than $\sin(x)$), it's zeros are all the positive and negative non-zero multiples of $\pi$. So we can write $\sin(x)$ as;
    \[\frac{\sin(x)}{x}=c\prod_{n=1}^{\infty}(1-\frac{x^2}{\pi^2n^2}).\]
Now, to solve for c, we will take the limit as x approaches 0;
\[\lim_{x\rightarrow 0}{\frac{\sin(x)}{x}}=1=c\prod_{n=1}^{\infty}(1-\frac{0^2}{\pi^2n^2})=c\]
which shows $c=1$. Next, we multiply both sides by $x$;
\[\sin(x)=x\prod_{n=1}^{\infty}(1-\frac{x^2}{\pi^2n^2})\]
as claimed.
\end{proof}
First, we will expand the first term in the infinite product of the sine function
\begin{equation}
    \sin(x)=x\prod_{n=1}^{\infty}(1-\frac{x^2}{\pi^2n^2})=x-x^3\sum_{n=1}^{\infty}\frac{1}{\pi^2n^2}….
\end{equation}
because a function cannot have 2 Taylor expansions, the terms must be equal. so;
\[\sin(x)=x\prod_{n=1}^{\infty}(1-\frac{x^2}{\pi^2n^2})=x-x^3\sum_{n=1}^{\infty}\frac{1}{\pi^2n^2}=x-\frac{x^3}{6}….\]
Comparing the $x^3$ terms;
\[\frac{x^3}{6}=\sum_{n=1}^{\infty}\frac{1}{\pi^2n^2}\]
or

\[\frac{\pi^2}{6}=\sum_{n=1}^{\infty}\frac{1}{n^2}\]
as claimed.
\end{proof}
\section{More about the Basel problem}
Mathematicians originally knew that the sum
    \[\sum_{n=1}^{\infty}\frac{1}{n}\]
diverges to infinity at a very slow rate (logarithmic growth). A lot of mathematicians attempted to sum the similar sum
    \[\sum_{n=1}^{\infty}\frac{1}{n^2}\]
but it was only formally posed by Pietro Mengoli as a challenge in 1650. Euler generalized his solution to
    \[\zeta (2n)={\frac {(-1)^{n+1}B_{2n}(2\pi )^{2n}}{2(2n)!}}\]
where $B_{n}$ is the nth Bernoulli number. Euler also included his famous Euler product;
    \[\zeta(s)^{-1}=\prod_{p\in\mathbb{P}}(1-\frac{1}{p^s})\]
a more analytic result than a summation result (see \cite{RS19}). Euler included a lot about $\zeta(2n)$, but never managed to get the value of $\zeta(3)$ in closed form. Euler is now well known for his work in computation of the Riemann Zeta function, and was very good considering his level of technology he had in 1735.
\section{Conclusion}
This is a truly beautiful result from Euler, and his way of doing so is very elegant and short.
\section{Acknowledgments}
Thanks to Bill Gosper for mentoring me and Simon Rubinstein-Salzedo for helping me write this (Bill has a proof of the Basel problem at \cite{Gos99}).
\bibliographystyle{alpha}
\bibliography{biblio}
\end{document}