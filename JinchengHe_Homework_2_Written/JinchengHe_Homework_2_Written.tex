\documentclass{article}
\usepackage[utf8]{inputenc}
\usepackage[english]{babel}
\usepackage[margin=1in]{geometry}

\usepackage{fancyhdr}
\usepackage{extramarks}
\usepackage{amsmath}
\usepackage{amsthm}
\usepackage{amsfonts}
\usepackage{tikz}
\usepackage[plain]{algorithm}
\usepackage{algpseudocode}
\usepackage{arydshln}
\usepackage{mathtools}
\usepackage{cases}
\usepackage{listings}
\usepackage[numbered]{mcode}
\usepackage{booktabs}
\usepackage{graphicx}
\usepackage{subfigure}

\usepackage{blindtext}
\usepackage{amssymb}
\usepackage{hyperref}
\hypersetup{
    colorlinks=true,
    linkcolor=blue,
    filecolor=magenta,
    urlcolor=cyan,
}

\urlstyle{same}

%\newtheorem{theorem}{Theorem}
\newtheorem{theorem}{Theorem}[section]
\newtheorem{corollary}{Corollary}[theorem]
\newtheorem{lemma}[theorem]{Lemma}
\theoremstyle{remark}
\newtheorem*{remark}{Remark}

\theoremstyle{definition}
\newtheorem{definition}{Definition}[section]

\title{CS 250 - Computer Architecture \\ Homework 2 Written}
\author{Jincheng He Email: jincheng.he@dukekunshan.edu.cn}
\date{October 11, 2021}

\begin{document}

    \maketitle


    \section{Q1 MIPS Instruction Set}
    \begin{enumerate}
        \item[(a)] What MIPS instruction is this? 0x022a4822

        The binary representation is: 0000 0010 0010 1010 0100 1000 0010 0010. So it should be 000000 10001 01010 01001 00000 100010. So Rs is \$s1, Rt is \$t2, Rd is \$t1.
        Based on Sh and Func, we can know it is sub.

        So the MIPS instrunction should be: sub \$t1, \$s1, \$t3.
        \item[(b)] What is the binary representation of this instruction lw \$t0, 8(\$t2)?

        The binary should be 100011 Rs Rt Immed. Rs is \$t2, which is 01010, Rt is \$t0, which is 01000, Immed is 8, which is 0000000000001000. So the binary representation should be: 100011 01010 01000 0000000000001000.
    \end{enumerate}


\end{document}